\documentclass{article}
\usepackage{amsfonts,amssymb,amsmath}
\usepackage[margin=1in]{geometry}

\newcommand{\qed}{\hfill\ensuremath{\blacksquare}}

\title{Solutions to Some Exercises}
\author{Geoffrey Matthews}
\date{\today}

\begin{document}

\maketitle

{\bf Note:  These problems are all solved in the book.  I include their
solutions here to demonstrate how to typeset them in \LaTeX.}

\begin{description}

\item[Chapter 1 Exercises]
\item[Section 1.1]
\item[1.] $\{5x-1:x\in\mathbb{Z}\} = \{...,-11,-6,-1,4,9,14,19,24,29,...\}$
\item[13.] $\{x\in\mathbb{Z}:|6x|<5\} = \{0\}$

\item[Section 1.2]
\item[1.]
  \begin{description}
  \item[(a)] $A\times B = \{(1,a),(1,c),(2,a)(2,c),(3,a),(3,c),(4,a),(4,c)\}$
  \end{description}

\item[Section 1.4]
\item[A.] Find the indicated sets.
\begin{description}
\item[3.] $\mathcal{P}(\{\{a,b\},\{c\}\}) = \{\emptyset, \{\{a,b\}\}, \{\{c\}\}, \{\{a,b\},\{c\}\}\}$
\end{description}

\item[B.] Suppose that $|A|=m$ and $|B|=n$. Find the indicated cardinalities.
\begin{description}
\item[13.] $|\mathcal{P}(\mathcal{P}(\mathcal{P}(A)))| = 2^{(2^{(2^m)})}$
\item[15.] $|\mathcal{P}(A\times B)| = 2^{mn}$
\end{description}

\item[Section 1.5]
\item[3.] Suppose $A=\{0,1\}$ and $B=\{1,2\}$.  Find:
\begin{description}
\item[(a)] $A\cup B = \{1,3,4,5,6,7,8,9\}$
\item[(b)] $A\cap B = \{4,6\}$
\end{description}


\item[Section 1.6]
\item[1.] Suppose $A=\{4,3,6,7,1,9\}$ and $B=\{5,6,8,4\}$ have the
  universal set $U=\{n\in\mathbb{Z}:0\leq n \leq 10\}$
  \begin{description}
  \item $\overline{\overline{A} \cap B} = \{0,1,2,3,4,6,7,9,10\}$
  \end{description}

\item[Section 1.8]
\item[5(a)] $\bigcup_{i\in\mathbb{N}} [i,i+1] = [1,\infty)$ or:
  \[\bigcup_{i\in\mathbb{N}} [i,i+1] = [1,\infty)\]

  \item[5(b)] $\bigcap_{i\in\mathbb{N}} [i,i+1] = \emptyset$ or:
    \[\bigcap_{i\in\mathbb{N}} [i,i+1] = \emptyset\]


\item[Chapter 2 Exercises]

\item[Section 2.2]  Express each statement as one of the forms $P\wedge Q$,
  $P\vee Q$, or $\sim P$.  (I will also accept $\neg P$.)
\item [9.] $x\in A-B$

  $(x\in A) \wedge \neg(x\in B)$

\item[Section 2.5]

\item[5.] Write a truth table for $(P\wedge \neg P)\vee Q$
  
  \begin{tabular}{|c|c||c||c|}\hline
    $P$  &   $Q$  & $(P\wedge \neg P)$  &   $(P\wedge \neg P)\vee Q$ \\\hline\hline
    $T$  & $T$  &  $F$  &  {\bf T} \\\hline
    $T$  & $F$  &  $F$  &  {\bf F} \\\hline
    $F$  & $T$  &  $F$  &  {\bf T} \\\hline
    $F$  & $F$  &  $F$  &  {\bf F} \\\hline
  \end{tabular}
  
\item[Chapter 3 Exercises]
\item[Section 3.3]
\item[1.] Suppose a set $A$ has 37 elements.  How may subsets of $A$
  have 10 elements?  How many subsets have 30 elements?  How many have
  0 elements?

  Answers:  ${37 \choose 10} = 348,330,136$ ; ${37 \choose 30} =
  10,295,472$; $ {37 \choose 0} = 1$.

\item[Chapter 4 Exercises]
\item[7.]  Suppose $a,b\in\mathbb{Z}$.  If $a \mid b$, then $a^2 \mid
    b^2$.

    {\em Proof.}  Suppose $a \mid b$.

    By definition of divisibility, this means $b=ac$ for some integer
    $c$.

    Squaring both sides of this equation produces $b^2=a^2c^2$.

    Then $b^2=a^2d$, where $d=c^2\in\mathbb{Z}$.

    By definition of divisibility, this means $a^2 \mid b^2$.
    \qed

\item[Chapter 5 Exercises]

\item[9. Proposition]  Suppose $n\in\mathbb{Z}$.  If $3 \nmid n^2$,
  then $3 \nmid n$.

  {\em Proof.}  (Contrapositive) Suppose it is not the case that
  $3\nmid n$, so $3 \mid n$.  This means that $n=3a$ for some integer
  $a$.  Consequently $n^2 = 9a^2$, from which we get $n^2=3(3a^2)$.
  This shows that there is an integer $b=3a^2$ for which $n^2=3b$,
  which means $3\mid n^2$.  Therefore it is not the case that $3\nmid
  n^2$.  
  \qed

%  \newcommand{\mod}[1]{\ (\mbox{mod}\ #1)}
  
 \item[21. Proposition]  Let $a,b\in\mathbb{Z}$ and $n\in\mathbb{N}$.
   If $a\equiv b \mod{n}$, then $a^3\equiv b^3\mod{n}$.

   {\em Proof.}  (Direct) Suppose $a\equiv b\mod{n}$.  This means
   $n\mid(a-b)$, so there is an integer $c$ for which $a-b=nc$.  Then:
   \begin{align*}
     a-b &= nc\\
     (a-b)(a^2 + ab+b^2) &= nc(a^2 + ab + b^2)\\
     a^3 + a^2b + ab^2 -ba^2 -ab^2 - b^3 &= nc(a^2+ ab + b^2)\\
     a^3 -b^3  &= nc(a^2+ ab + b^2).
   \end{align*}
   Since $a^2 + ab + b^2\in\mathbb{Z}$, the equation $a^3-b^3=nc(a^2+
   ab + b^2)$ implies $n\mid (a^3-b^3)$, and therefore $a^3\equiv
   b^3\mod{n}$.     \qed

\item[Chapter 9 Exercises]
\item[27.] The equation $x^2=2^x$ has three real solutions.

  {\em Proof.}  By inspection, $x=2$ and $x=4$ are two solutions of
  this equation.  But there is a third solution.  Let $m$ be the real
  number for which $m2^m = \frac{1}{2}$.  Then negative number $x=-2m$
  is a solution, as follows.
  \[
  x^2 = (-2m)^2
      = 4m^2
      = 4\left( \frac{m2^m}{2^m} \right)^2
      = 4\left(\frac{\frac{1}{2}}{2^m}\right)^2
      = \frac{1}{2^{2m}}
      = 2^{-2m}
      = 2^x
      \]

\item[Chapter 10 Exercises]

\item[1.] For every integer $n\in\mathbb{N}$, it follows that
  \[
  1 + 2 + 3 + 4 + \ldots + n  = \frac{n^2 + n}{2}
  \]
  or
  \[
  \sum_{i=1}^{n} i  = \frac{n^2 + n}{2}
  \]
  {\bf In this proof I use the second notation.  The book shows the
  solution in the first notation.}

  {\em Proof.}  we will prove this with mathematical induction.

  (1) Observe that if $n=1$, this statement is $1=\frac{1^2 + 1}{2}$,
  which is obviously true.

  (2) Consider any integer $k \geq 1$.  We must show that $S_k$
  implies $S_{k+1}$.  In other words, we must show that if
  \[
  \sum_{i=1}^{k} i = \frac{k^2+k}{2}
  \]
  is true, then
  \[
  \sum_{i=1}^{k+1} i = \frac{(k+1)^2+(k+1)}{2}
  \]
  is also true.  We use direct proof.

  Suppose $k\geq 1$ and 
  \[
  \sum_{i=1}^{k} i = \frac{k^2+k}{2}
  \]
  We observe that
  \begin{align*}
    \sum_{i=1}^{k+1} i
    &= \sum_{i=1}^{k} i + (k+1) &\text{(isolating the last term in the sum)} \\
    &= \frac{k^2+k}{2} + (k+1) & \text{(by the inductive hypothesis)}\\
    &= \frac{k^2 + k + 2(k+2)}{2} \\
    &= \frac{k^2 + 2k + 1 + k + 1}{2}\\
    &= \frac{(k+1)^2 + (k+1)}{2}
  \end{align*}
  Therefore we have shown that 
  \[
  \sum_{i=1}^{k+1} i = \frac{(k+1)^2+(k+1)}{2}
  \]

\end{description}


\end{document}
