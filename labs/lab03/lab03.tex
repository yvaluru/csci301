\documentclass{article}
\pagestyle{empty}
\usepackage[margin=1in]{geometry}
\usepackage{fancyvrb,amsmath}
\usepackage{multicol}

\newcommand{\set}[1]{\ensuremath{\{#1\}}}
\newcommand{\power}[1]{\ensuremath{\mathcal{P}(#1)}}

\title{CSCI 301, Lab \# 3}
\author{Winter 2018}
\date{}
\begin{document}

\maketitle
%\setlength{\columnsep}{2em}
%\begin{multicols}{2}


\paragraph{Goal:} The purpose of this lab is write some  code 
 using lists.  All procedures should be written using {\tt car} and
 {\tt cdr} and recursion to traverse lists.  Do not use {\tt map},
 write out the recursive function needed to get the job done.  You may
 use {\tt append} (see below).

\paragraph{Due:} Your program, named {\tt lab03.rkt}, must be submitted to
  Canvas before midnight, Monday, Feb 5.

  \paragraph{Unit tests:}
  At a minimum, your program must pass the unit tests found in the
  file {\tt lab03-test.rkt}.  Place this file in the same folder
  as your program, and run it;  there should be no output.

\paragraph{Finding subsets:}  Suppose we want to procedurally find all
the subsets of a given set, $A=\set{1,2,3}$, the power set,
$\mathcal{P}(A)$.

One way to think of this is to break the subsets
into two groups by picking a single element of $A$, for example, 1,
and dividing \power{A} into subsets that have 1 in them, and subsets that
don't.

Call the ones that don't have 1 in them $A_0$ and the ones
that do, $A_1$.  In our example, we have:
\begin{align*}
A_0 &= \set{\emptyset,\set{2},\set{3},\set{2,3}}\\
A_1 &= \set{\set{1},\set{1,2},\set{1,3},\set{1,2,3}}
\end{align*}
Note that the sets in $A_1$ are just the sets in $A_0$ with a 1 added
to them.

The power set is just the union of these two:
\[
\power{A} = A_0 \cup A_1
\]
Note that we now have a recursive definition of the power set:
\[
\power{A} = \left\{\begin{array}{ll}
\set{\emptyset} & \mbox{if $A=\emptyset$}\\
A_0 \cup A_1 & \mbox{otherwise}
\end{array}\right.
\]
where $A_0$ are all the subsets of a set without one of the elements
of $A$, and $A_1$ are all the subsets with that element.

\paragraph{Program:} We'll use the above ideas to write a Scheme
program to create sublists of a list.
{\small
\begin{Verbatim}[frame=single]
> (sublists '())
'(())
> (sublists '(1 2))
'(() (2) (1) (1 2))
> (sublists '(1 2 3))
'(() (3) (2) (2 3) (1) (1 3) (1 2) (1 2 3))
> (sublists '(1 2 3 4))
'(() (4) (3) (3 4) (2) (2 4) (2 3) (2 3 4) (1) (1 4) (1 3) (1 3 4) (1 2) (1 2 4) (1 2 3) (1 2 3 4))
\end{Verbatim}
}
This procedure should be easy, given the above insights.  If the list
is empty, the value is simple.  If the list {\tt ls} is not empty,
then find the sublists of {\tt (cdr ls)}.  Save this list in a local
variable.  (This represents the set $A_0$.) Call another procedure to
add the {\tt (car ls)} to each of the lists in this set.  Call this
procedure {\tt distribute}.  It works like this:
\begin{Verbatim}[frame=single]
> (distribute 7 '((1 2 3) (4 5) (1 1 1)))
'((7 1 2 3) (7 4 5) (7 1 1 1))
\end{Verbatim}
Note that the order returned by {\tt distribute } will depend on
whether you use tail-recursion.  (Why?)

Now just append the two lists to get the final result.

\paragraph{Sorting the results:}
The results we get are not very satisfying as regards their order.
Clearly, the second order here is better than the
first:\footnote{Note: although I call this new procedure ``{\tt
    sublists},'' it only sorts the lists.  It does not remove
  duplicates, {\em etc.}}
{\small
  \begin{Verbatim}[frame=single]
> (sublists '(1 2 3 4))
'(() (4) (3) (3 4) (2) (2 4) (2 3) (2 3 4) (1) (1 4) (1 3) (1 3 4) (1 2) (1 2 4) (1 2 3) (1 2 3 4))
> (subsets '(1 2 3 4))
'(() (1) (2) (3) (4) (1 2) (1 3) (1 4) (2 3) (2 4) (3 4) (1 2 3) (1 2 4) (1 3 4) (2 3 4) (1 2 3 4))
\end{Verbatim}
}
We can get this simply by sorting the results from the {\tt sublists}
procedure.  Scheme has a builtin sorting function, which takes a
two-place boolean operator to decide how to sort:
\begin{Verbatim}[frame=single]
> (sort '(3 5 2 9 1) <)
'(1 2 3 5 9)
> (sort '(3 5 2 9 1) >)
'(9 5 3 2 1)
\end{Verbatim}
So all you have to do is write some two-place boolean operators that,
first, sorts by length of the list, and then, within lists of the same
length, sorts by elements.  For example, the function {\tt
  element-ordered?} returns \verb|#t| if the lists are the same, or
the first differing element is smaller in the first list, and
\verb|#f| otherwise:
\begin{Verbatim}[frame=single]
> (element-ordered? '(4 7 9) '(4 7 9))
#t
> (element-ordered? '(1 3 5) '(1 3 4))
#f
> (element-ordered? '(1 3 5 8) '(1 3 6 7))
#t
\end{Verbatim}
And another function, {\tt length-ordered?}, which returns \verb|#t|
if the first list is shorter, \verb|#f| if the first list is
longer, and the result of {\tt element-ordered?} if they are the same
length. 

Putting these together gives such spectacular results as this:
{\scriptsize
  \begin{Verbatim}[frame=single]
> (subsets '(1 2 3 4 5))
'(()
  (1)
  (2)
  (3)
  (4)
  (5)
  (1 2)
  (1 3)
  (1 4)
  (1 5)
  (2 3)
  (2 4)
  (2 5)
  (3 4)
  (3 5)
  (4 5)
  (1 2 3)
  (1 2 4)
  (1 2 5)
  (1 3 4)
  (1 3 5)
  (1 4 5)
  (2 3 4)
  (2 3 5)
  (2 4 5)
  (3 4 5)
  (1 2 3 4)
  (1 2 3 5)
  (1 2 4 5)
  (1 3 4 5)
  (2 3 4 5)
  (1 2 3 4 5))
\end{Verbatim}
}
\end{document}

\end
